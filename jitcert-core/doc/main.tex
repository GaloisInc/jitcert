
% \documentclass[format=acmsmall, review=false, screen=true, pdftex,hyperref,svgnames]{acmart}
\documentclass{article}

\usepackage[hidelinks]{hyperref}
\usepackage{cleveref}
\usepackage{graphicx}
\usepackage{xcolor}
\usepackage{xspace}

% UPDATES
%% X Make sure figure code is up to date.
%% X New solution node requirements/sentences.
%% X New queries
%%   Describe web UI
%% X Logic nodes
%%% X Normalizing GSNs
%% X Add figure 4 as pattern by turning it into a combinator?

% X Embedded DSL in Haskell
% X Builder monad to create GSNs.
% X Example of creating nodes and edges.
% X Correct by construction, enforced by Haskell's type system
%% X Explain rules
% X Sentences
%%   Reference to other nodes.
%%   predicates
% X Noun/verb context nodes
% X Node policies
%   Setting values
%%   Values for noun context nodes
%%   Evidence for solution nodes
%   Checks for wellformedness
%%   Scoping, etc
%   Patterns
%%   Combinators
%%   Queries

% Rendering GSNs.

% Mapping to boolean logic/expressions
% Compute if overall GSN is true.
% Incremental checking once values in the GSN environment have changed.

\newcommand{\jitcert}{JitCert\xspace}


\usepackage{listings}
\usepackage{amssymb}
% uncomment next line to restore colors
% \def\withcolor{}


\newsavebox\MBox
\newcommand\Cline[2][red]{{\sbox\MBox{$#2$}%
  \rlap{\usebox\MBox}\color{#1}\rule[-1.2\dp\MBox]{\wd\MBox}{0.5pt}}}

\ifdefined\withcolor
	 \definecolor{haskellblue}{rgb}{0.0, 0.0, 1.0}
	 \definecolor{haskellstr}{rgb}{0.2, 0.2, 0.6}
	 \definecolor{haskellred}{rgb}{1.0, 0.0, 0.0}
  \definecolor{gray_ulisses}{gray}{0.55}
  \definecolor{castanho_ulisses}{rgb}{0.71,0.33,0.14}
  \definecolor{preto_ulisses}{rgb}{0.41,0.20,0.04}
  \definecolor{green_ulises}{rgb}{0.2,0.75,0}
\else
	\definecolor{haskellblue}{gray}{0.1}
	\definecolor{haskellstr}{gray}{0.1}
	\definecolor{haskellred}{gray}{0.1}
	\definecolor{gray_ulisses}{gray}{0.1}
	\definecolor{castanho_ulisses}{gray}{0.1}
	\definecolor{preto_ulisses}{gray}{0.1}
	\definecolor{green_ulisses}{gray}{0.1}
\fi

% RJ: I don't like the colors. Too confusing.
%
% \definecolor{lcolor}{rgb}{0.0, 0.0, 1.0}
% \definecolor{lappcolor}{rgb}{1.0, 0.0, 0.0}
% \definecolor{lappascolor}{rgb}{0.0, 1.0, 0.0}

\definecolor{lcolor}{gray}{0.0}
\definecolor{lappcolor}{gray}{0.0}
\definecolor{lappascolor}{gray}{0.0}

\def\codesize{\small}

\definecolor{usercolor}{RGB}{47, 46, 51}
\definecolor{liocolor}{RGB}{10, 55, 104}
\definecolor{yesodcolor}{RGB}{11, 104, 51}
\definecolor{dbcolor}{RGB}{47, 46, 51}

% \newcommand\showfocus[1]{\color{red}{#1}}
\newcommand\showyesod[1]{\color{yesodcolor}{#1}}
\newcommand\showlio[1]{\color{liocolor}{#1}}
% \newcommand\showfocus[1]{\underbar{#1}}
% \newcommand\showfocus[1]{\color{purple}{\textbf{#1}}}

\lstdefinelanguage{HaskellUlisses} {
	basicstyle=\ttfamily\small,
	moredelim=[is][\showyesod]{\*}{\*},
	moredelim=[is][\showlio]{\^}{\^},
	sensitive=true,
	%% morecomment=[l][\color{gray_ulisses}\ttfamily\itshape\codesize]{--},
	%% morecomment=[s][\color{gray_ulisses}\ttfamily\itshape\codesize]{-}{-},
	morecomment=[l][\color{gray_ulisses}\ttfamily\itshape\codesize]{--},
	morestring=[b]",
	stringstyle=\color{haskellstr},
	basewidth={0.53em},
	showstringspaces=false,
	numberstyle=\codesize,
	numberblanklines=true,
	showspaces=false,
	breaklines=true,
	showtabs=false,
  %% whitespace hackery
  %% lineskip= -2pt,
  %% aboveskip=1pt,
  %% belowskip=1pt,
  literate={ {quals}{{$\mathbb{Q}$}}1
             {iquals}{{$\mathbb{Q}$}}2
             {ltsolzero}{{$A_0$}}2
             {band}{{$\textbf{\texttt{and}}$}}2
             {->}{{$\rightarrow$}}2
             {lambda}{{$\lambda$}}2
             {<-}{{$\leftarrow$}}1
             {monotonic}{{monotonic}}9
             % {not}{{$\neg\!\!\!$}}2
             {===}{{$\equiv$}}2
             {ς}{{$\varsigma$}}1
             {ε}{{$\epsilon$}}1
             {φ}{{$\phi$}}1
             {canFlowTo}{{$\sqsubseteq$}}1
             {cannotFlowTo}{{$\not\sqsubseteq$}}1
%             {&&}{{$\land$}}1
             {meet}{{$\sqcap$}}1
             {jjoin}{{join}}4
             {join}{{$\sqcup$}}1
             {bot}{{$\perp$}}1
             {top}{{$\top$}}1
             {=>}{{$\Rightarrow$}}1
             {<=>}{{$\Leftrightarrow$}}1
             {GType}{{\tilde{\mathit{\texttt{Type}}}}}4
           },
	emph=
	{[1]
		FilePath,IOError,abs,acos,acosh,all,and,any,appendFile,approxRational,asTypeOf,asin,
		asinh,atan,atan2,atanh,basicIORun,break,catch,ceiling,chr,compare,concat,concatMap,
		const,cos,cosh,curry,cycle,decodeFloat,denominator,digitToInt,div,divMod,drop,
		dropWhile,either,elem,encodeFloat,enumFrom,enumFromThen,enumFromThenTo,enumFromTo,
		error,even,exp,exponent,fail,mapMaybe,filter,flip,floatDigits,floatRadix,floatRange,floor,
		fmap,foldl,foldl1,foldr,foldr1,fromDouble,fromEnum,fromInt,fromInteger,fromIntegral,
		fromRational,fst,gcd,getChar,getContents,getLine,head,id,inRange,index,init,intToDigit,
		interact,ioError,isAlpha,isAlphaNum,isAscii,isControl,isDenormalized,isDigit,isHexDigit,
		isIEEE,isInfinite,isLower,isNaN,isNegativeZero,isOctDigit,isPrint,isSpace,isUpper,iterate,
		last,lcm,length,lex,lexDigits,lexLitChar,lines,log,logBase,lookup,map,mapM,mapM_,max,
		maxBound,posMax,negMax,maximum,maybe,min,minBound,minimum,mod,negate,not,notElem,null,numerator,odd,
		or,ord,pi,pred,primExitWith,print,product,properFraction,putChar,putStr,putStrLn,quot,
		quotRem,range,rangeSize,read,readDec,readFile,readFloat,readHex,readIO,readInt,readList,readLitChar,
		readLn,readOct,readParen,readSigned,reads,readsPrec,realToFrac,recip,rem,repeat,replicate,return,
		round,scaleFloat,scanl,scanl1,scanr,scanr1,seq,sequence,sequence_,show,showChar,showInt,
		showList,showLitChar,showParen,showSigned,showString,shows,showsPrec,significand,signum,sin,
		sinh,snd,span,splitAt,sqrt,subtract,succ,sum,tail,take,takeWhile,tan,tanh,threadToIOResult,toEnum,
		toInt,toInteger,toLower,toRational,toUpper,truncate,uncurry,undefined,unlines,until,unwords,unzip,
		unzip3,userError,words,writeFile,zip,zip3,zipWith,zipWith3,listArray,doParse,empty,for,initTo,
        assert,compose,checkGE,maxEvens,empty,create,get,set,initialize,idVec,fastFib,fibMemo,
        ex1,ex2,ex3,incr,inc,dec,isPos,positives,find,insert,len,size,union,fromList,initUpto,trim,
        insertSort,decsort,qsort,reverse,append,upperCase, ifM, whileM, get, decrM, diff,
        project, select, leq, elts, keys, dkeys, dfun, addKey, pTrue, emptyRD, rFalse,
        	dom, rng, isI, isD, isS, movie1, movie2,  toI, toS, toD, good_titles, runState, ret,
        	update, getCtr, setCtr, ctr, rdCtr, wrCtr, ifTest, whileTest, posCtr, zeroCtr, decr, decCtr,
        	pread , pwrite , plookup , pcontents, pcreateF , pcreateFP, pcreateD, active, caps, pset, eqP,
        	write, contents, alloc, derivP, copyP, createDir, store, copyRec, copySpec,
        	forM_, when, flookup, fread, createDir, pcreateFile, isFile, copyFrame, ?
	},
	emphstyle={[1]\color{haskellblue}},
	emph=
	{[2] Eq, Program, Label, 
	},
	emphstyle={[2]\color{castanho_ulisses}},
	emph=
	{[3]
		case,class,data,deriving,do,else,if,import,in,infixl,infixr,instance,let,
		module,of,primitive,then,refinement,type,where,forall,bound,and,
		% otherwise,
		measure,reflect,predicate, assume, return
	},
	emphstyle={[3]\color{preto_ulisses}\textbf},
	emph=
	{[4]
		quot,rem,div,mod,elem,notElem,seq
	},
	emphstyle={[4]\color{castanho_ulisses}\textbf},
	emph=
	{[5]
		EQ,GT,LT,Left,Right
		%, False, True, Just, Nothing
	},
	emphstyle={[5]\color{preto_ulisses}\textbf},
	emph=
	{[6]
	    axiomatize, measure, inline, return
	},
	emphstyle={[6]\color{lcolor}}
}

%%%ORIG
%%%\lstnewenvironment{code}
%%%{\textbf{Haskell Code} \hspace{1cm} \hrulefill \lstset{language=HaskellUlisses}}
%%%{\hrule\smallskip}

%V1
%\lstnewenvironment{code}
%{\smallskip \lstset{language=HaskellUlisses}}
%{\smallskip}

\lstnewenvironment{code}
{\lstset{language=HaskellUlisses}}
{}

\lstnewenvironment{mcode}
{\lstset{language=HaskellUlisses,columns=fullflexible,keepspaces,mathescape}}
{}

\lstMakeShortInline[language=HaskellUlisses,mathescape,keepspaces,mathescape,basicstyle=\ttfamily\small,breakatwhitespace]@


\lstdefinelanguage{Pseudo} {
	basicstyle=\ttfamily\codesize,
	sensitive=true,
  mathescape=true,
	morecomment=[l][\color{gray_ulisses}\ttfamily\codesize]{--},
	morecomment=[s][\color{gray_ulisses}\ttfamily\codesize]{\{-}{-\}},
	morestring=[b]",
	showstringspaces=false,
	numberstyle=\codesize,
	numberblanklines=true,
	showspaces=false,
	breaklines=true,
	showtabs=false
}

\lstset{basicstyle=\ttfamily\footnotesize,breaklines=true}

\begin{document}
\title{\jitcert Documentation}

% \author{}

\maketitle

\section{Introduction}

\jitcert is a domain specific language (DSL) for defining, rendering, checking, and querying 
assurance case arguments in Goal Structuring Notation (GSN) \cite{kelly2004goal}. 
\jitcert is implemented as an embedded DSL in Haskell. 

\section{Defining GSNs}

The \jitcert library provides a \texttt{Builder} monad to construct GSNs (available in the \texttt{JitCert.GSN.Builder} module). 
Each node type (goal, context, solution, strategy, assumption, justification) has a function to create a new node and insert the node into the GSN. 
For example, we can create a goal node with the \texttt{goal} function:
\begin{mcode}
g <- goal PolicyAnd [SFText "Top level goal"]
\end{mcode}
This creates a goal node with a string label \texttt{"Top level goal"}. 
In addition, this node has a conjunction policy that requires all child nodes to be \texttt{true}. 
Policies are described in detail in \Cref{sec:policies}. 

We can create a solution node for the goal node with the \texttt{solution} and \texttt{addEdge} functions:
% s <- solution @() [SFText "Solution node"]
\begin{mcode}
s <- solution SolutionNode
addEdge g s
\end{mcode}
First we create a solution node that holds a trivial \texttt{SolutionNode} value. % with a string label. 
% This solution node does not require any type of evidence, so we give it unit (\texttt{()}) as the evidence type. 
% Evidence types are documented in \Cref{subsec:evidence}. 
Next we add an edge between the goal node, \texttt{g}, and the solution node, \texttt{s}.
This creates the GSN shown in \Cref{fig:fig1}. 
% Shapes of GSN nodes are rendered according to their types, as is convention.

Solution nodes in JitCert hold values that describe what type of evidence is required to satisfy the node and how to visually display that node. 
The \texttt{SolutionType} typeclass is used to specify the evidence type and text rendering for a node. 
\begin{mcode}
class SolutionType s where
    type SolutionEvidence s

    renderSolution :: s -> Sentence
\end{mcode}
% Should we use a different example that's clearer? Passing test cases and test results?
\texttt{SolutionNode} is a simple type for the solution node in \Cref{fig:fig1}.
This solution node does not require any type of evidence, so we give it unit (\texttt{()}) as the evidence type. 
The node's description is always rendered as \texttt{"Solution node"}. 
\begin{mcode}
data SolutionNode = SolutionNode

instance SolutionType SolutionNode where
    type SolutionEvidence SolutionNode = ()

    renderSolution SolutionNode = [SFText "Solution node"]
\end{mcode}

\begin{figure}
\centering
\includegraphics[scale=0.5]{img/figure1.png}
\caption{Simple GSN example.}
\label{fig:fig1}
\end{figure}

\subsection{Correct GSN structure}

GSN places the following constraints on the structure of nodes. 
% The following are the constraints on nodes. 
%
\begin{itemize}
\item Goal nodes may have children with any type of node.
\item Strategy nodes may only have children of goal, context, assumption, and justification nodes.
\item Any other node combinations are forbidden. 
\end{itemize}
%
\jitcert uses Haskell's type system
to enforce these constraints. 
This results in \jitcert GSNs that are correct by construction.

% \subsection{Context nodes}

\subsection{Node labels} % references}

Nodes in \jitcert GSNs are labeled with sentences, which are lists of sentence fragments. 
When rendering a node's label, sentence fragments are concatenated with spacing in between. 
Here is the type definition for a sentence fragment (module \texttt{JitCert.GSN.Types}). 
\begin{mcode}
data SentenceFragment =
    SFText T.Text
  | forall c . RenderContext c => SFContext (Node c Context)
  | forall p . Property p => SFProperty p
\end{mcode} % JP: Rename RenderContext to ContextValue?
As previously seen, a sentence fragment can be a string (ex, \texttt{SFText "Top level goal"}), which directly renders as the string.

Sentence fragments can also contain references to context nodes by using the \texttt{SFContext} constructor.  
Referenced context nodes must be instances of the \texttt{RenderContext} typeclass so that the GSN can display the node reference as part of the sentence. 
% Sentence fragments must only reference noun contexts. 
In addition to visually displaying the referenced node, 
node references establish a dependence of the current node to the referenced node. 
If the referenced node's value changes, the current node is invalidated and needs to be recertified. 

% TODO: An example would probably help.

The final sentence fragment constructor is \texttt{SFProperty} that can hold any type that implements the \texttt{Property} typeclass. 
Properties are typically data types representing predicates. 
For example, one might define a \texttt{NoIO} type as a property to indicate a piece of software does not perform any input or output operations.
Properties define how they should be rendered. 
Properties may also contain references to other context nodes, which again invalidate the current node if the referenced node's value changes.

\subsection{Context nodes}
\label{subsec:contextnodes}

There are two kinds of context nodes: noun context nodes and verb context nodes. 

\paragraph{Noun context nodes.}
Noun context nodes represent an object that the GSN is reasoning about. 
Noun context nodes introduce a variable into scope for the context node's parent and the parent's descendents. 
For example, if a GSN is arguing that a software program is safe, the GSN would introduce the software program with a context node. 
\begin{mcode}
p <- context @SoftwareProgram "p" Nothing
\end{mcode}
This creates a context node % that can hold a
of a 
\texttt{SoftwareProgram} with variable name \texttt{"p"} and no custom label. 
% 
Now we can create a goal node that introduces the software program context node and 
a context node for the test case that tests the program. 
\begin{mcode}
g <- goal PolicyAnd [SFContext p, SFProperty IsSafe]
addEdge g p

t <- context @Test "t" Nothing
addEdge g t
\end{mcode}
% Notice that the goal, \texttt{g}, 
Goal, \texttt{g}, argues that program, \texttt{p}, supports the \texttt{IsSafe} property. 
%
%
Finally, we add a solution node that supports the goal with test case, \texttt{t}.
\begin{mcode}
s <- solution (Tests t p IsSafe)
addEdge g s
\end{mcode}
% \begin{mcode}
% s <- solution @TestResult [SFText "Test that", SFContext p, SFText "is safe"]
% addEdge g s
% \end{mcode}
Both the goal node, \texttt{g}, and the solution node, \texttt{s}, 
have references to the context node for software program, \texttt{p}.
Solution node, \texttt{s}, also has a reference to the test context node, \texttt{t}, 
because the result of running a test case should be invalidated when the test case is changed.
The GSN for this example is shown in \Cref{fig:fig2}.



\begin{figure}
\centering
\includegraphics[scale=0.5]{img/figure2.png}
\caption{Example of GSN with references to context nodes.}
\label{fig:fig2}
\end{figure}


% Nodes that are referenced must be noun context nodes.
% If a node is referenced 

\paragraph{Verb context nodes.}
Verb context nodes describe a situation. 
A verb context node's parent argument only applies when the situation is true. 
For example, we can build an argument that a program is compatible with the following GSN. 
\begin{mcode}
p <- context @SoftwareProgram "p" Nothing

g <- goal PolicyAnd [SFContext p, SFText "is compatible"]
addEdge g p

g1 <- goal PolicyAnd [SFContext p, SFProperty UsesLinuxLibraries]
addEdge g g1

s <- solution (ManualInspection p UsesLinuxLibraries)
addEdge g1 s
\end{mcode}
This GSN supports the top level goal with a subgoal that the program uses Linux libraries. 
This subgoal is only relevant when the program is targeting Linux. 
We can encode this information with the \texttt{contextVerb} function. 
\begin{mcode}
c <- contextVerb "targets Linux" programTargetsLinux
addEdge g1 c

programTargetsLinux :: SoftwareProgram -> Bool
\end{mcode}
We also need to update the policy for goal \texttt{g1}. 
\begin{mcode}
g1 <- goal (PolicyWhen p c PolicyAnd) [SFContext p, SFProperty UsesLinuxLibraries]
\end{mcode}
The verb context node \texttt{c} indicates the parent goal \texttt{g1} only applies when the predicate \texttt{programTargetsLinux} is true for the program \texttt{p}.
This GSN is shown in \Cref{fig:fig3}. 


\begin{figure}
\centering
\includegraphics[scale=0.5]{img/figure3.png}
\caption{Example of GSN with references to context nodes.}
\label{fig:fig3}
\end{figure}

\subsection{Node policies}
\label{sec:policies}

Nodes can have different policies that specify how a node's argument depends on its children. 
Here is the type that specifies node policies. 
\begin{mcode}
data NodePolicy c =
    PolicyAnd
  | PolicyOr
  | PolicyNot (NodePolicy c)
  | PolicyWhen (Node c Context) (Node (c -> Bool) Context) (NodePolicy c)
  | PolicyForall (Node c Context) (Node [c] Context)
\end{mcode}
\texttt{PolicyAnd} means that a node is only valid when all of its children are valid. 
\texttt{PolicyOr} means that a node is valid when any of its children are valid. 
\texttt{PolicyNot} is the negation of the subpolicy (first argument) on its children. 
\texttt{PolicyWhen} is used with verb context nodes (second argument). 
The node's policy (third argument) only applies when the predicate from the verb context node is true for the referenced context node (first argument). 
Otherwise, it is trivially valid. 
An example was previously shown in \Cref{subsec:contextnodes}. 
\texttt{PolicyForall} takes two context nodes as arguments. 
The second node serves as a reference to a set of values, while the first node is introduced as a reference to each element in the set of values. 
The node is valid only when its children are valid for all elements in the set of values. 
% Example?

To help construct nodes with for all policies, \jitcert provides the \texttt{forall} function, which has the following type. 
\begin{mcode}
forall :: (RenderContext c, Monad m) => Variable -> Node [c] Context -> BuilderT m (Node c Context, NodePolicy c)
\end{mcode}
\texttt{forall} takes a variable name and the node that references the set of values. 
It returns an introduced node with a reference to the elements of the set of values and the for all policy that can be used in the creation of other nodes.


% Mapping to boolean expressions?

\subsection{Logic nodes}

There are times when a GSN node only conveys logical information and does not make a verbal argument. 
To support this situation, \jitcert offers logic nodes to support logical \emph{and}, \emph{or}, and \emph{not}. 
\emph{and} nodes are displayed as squares and require all child nodes to be valid.
\emph{or} nodes are displayed as diamonds and require any child node to be valid. 
\emph{not} nodes are displayed as triangles and its validity is the negation of whether the subpolicy is valid. 

Logic nodes can be constructed with the \texttt{andNode}, \texttt{orNode}, and \texttt{notNode} functions.
\begin{code}
andNode :: Monad m => BuilderT m (Node c Goal)
orNode  :: Monad m => BuilderT m (Node c Goal)
notNode :: Monad m => NodePolicy c -> BuilderT m (Node c Goal)
\end{code}

As an example, consider the situation where a strategy requires the first goal to hold, while neither the second and third goal should hold. 
This can be encoded by the following.
\begin{mcode}
s <- strategy PolicyAnd [SFText "Strategy 1"]

n1 <- andNode
addEdge s n1

n2 <- notNode PolicyOr
addEdge n1 n2

g1 <- goal PolicyAnd [SFText "Goal 1"]
addEdge n1 g1

g2 <- goal PolicyAnd [SFText "Goal 2"]
addEdge n2 g2

g3 <- goal PolicyAnd [SFText "Goal 3"]
addEdge n2 g3
\end{mcode}
This results in the GSN shown in \Cref{fig:logic}.

\begin{figure}
\centering
\includegraphics[scale=0.5]{img/figureLogic.png}
\caption{Logic nodes in a GSN.}
\label{fig:logic}
\end{figure}


\subsection{Normalizing GSNs}
\label{subsec:normalize}

There are times when it is not possible to visualize a GSN's policy or a policy goes against convention.
\jitcert aids in this situation by providing the \texttt{normalizeGSN} function to normalize a GSN. 
\begin{code}
normalizeGSN :: GSNGraph -> GSNGraph
\end{code}
By convention, sibling goal nodes imply a conjunction for the parent node. 
If the parent's policy is not a \texttt{PolicyAnd}, \texttt{normalizeGSN} will insert an intermediate logic node so that the policy is clear visually. 
Similarly, sibling strategy nodes imply disjunction, so \texttt{normalizeGSN} will insert an intermediate logic node when the parent's policy is not a \texttt{PolicyOr}. 

Consider \Cref{fig:logic} as an example. 
Node \texttt{n2} is a \emph{not} node with an \emph{or} subpolicy. 
Its children are goal nodes, which implies conjunction.
To visually indicate the \emph{or} subpolicy, \texttt{normalizeGSN} inserts a logical \emph{or} node (\Cref{fig:normalize}).


\begin{figure}
\centering
\includegraphics[scale=0.5]{img/figureNormalize.png}
\caption{Normalization of logic nodes in a GSN.}
\label{fig:normalize}
\end{figure}


\section{GSN Environments}

In addition to structuring an argument, it is important to be able to reason about the objects and evidences that are being argued about in a GSN. 
\jitcert supports this by allowing users to create an environment that assigns values to nodes in the graph. % in two ways. 
Specifically, values can be assigned for noun context nodes and solution nodes. 

\paragraph{Noun context nodes}

\begin{figure}
\centering
\includegraphics[scale=0.5]{img/figure4.png}
\caption{Example of GSN with references to context nodes.}
\label{fig:fig4}
\end{figure}

Consider the argument that a software function is correct if it passes its test.
\begin{mcode}
f <- context @Function "f" Nothing
t <- context @Test "t" Nothing

g <- goal PolicyAnd [SFContext f, SFText "is correct"]
addEdge g f
addEdge g t

s <- solution (Tests t f IsCorrect)
addEdge g s
\end{mcode}
Suppose we want to reason about the implementation of the fibonacci function. 
We can assign a value for the function \texttt{f} and its test \texttt{t} with \texttt{setContextValue}.
\begin{mcode}
setContextValue f (Function "fibonacci" "c9a0ad2c")
setContextValue t (Test "test_fibonacci" "63d1cc08")
\end{mcode}
This binds the fibonacci function to \texttt{f} and the fibonacci test to \texttt{t} in the environment. 
\Cref{fig:fig4} shows the GSN with its updated environment. 



% TODO: Nested environments

\paragraph{Solution nodes}
In \jitcert, solution nodes can be assigned evidence values in the environment. 
Continuing the previous example, solution node \texttt{s} takes an evidence of type \texttt{TestResult}. 
We can set the evidence value in the environment with the \texttt{setSolutionEvidence} function. 
\begin{mcode}
setSolutionEvidence s (TestResult True)
\end{mcode}
This gives solution \texttt{s} a passing test result as its evidence. 
% Types used as evidence for solutions must implement the \texttt{Evidence} typeclass. 

% Evidence type...
% typeclass
% setEvidence


\subsection{Nested environments}
\label{subsec:nestedenv}
There are situations when the value of one node may influence the values of other nodes. 
For example, if there are multiple functions, each function may have its own test case. 
To support this functionality, \jitcert provides the \texttt{setContextValuesWith} function to create nested environments. 
% 
% 
% We can create 
% 
% nested environments
Let's recreate the previous example, except check that a list of functions are all correct.
\begin{mcode}
fs <- context @[Function] "fs" Nothing
(f, p) <- forall "f" fs

gs <- goal p [SFContext f, SFText "is correct"]
addEdge gs fs

t <- context @Test "t" Nothing
addEdge gs t

s <- solution (Tests t f IsCorrect)
addEdge gs s
\end{mcode}
Now we can argue that two functions, fibonacci and foo, are correct.
\begin{code}
let functions = [ (Function "fibonacci" "c9a0ad2c", Test "test_fibonacci" "63d1cc08", TestResult True)
                , (Function "foo" "f9434869", Test "test_foo" "e71b685f", TestResult False)]
setContextValuesWith fs functions $ lambda(function, functionTest, functionTestResult) -> do
      setContextValue t functionTest
      setSolutionEvidence s functionTestResult

      return function
\end{code}
First, we create a list of functions with their associated test and test result. 
We pass this to \texttt{setContextValuesWith} and set the values for \texttt{t} and \texttt{s} in the nested environment. 
The value for \texttt{f} is set automatically. 
\Cref{fig:fig5} shows this GSN. 

\begin{figure}
\centering
\includegraphics[scale=0.5]{img/figure5.png}
\caption{Example of GSN with references to context nodes.}
\label{fig:fig5}
\end{figure}


\section{Well-formedness}

\jitcert's use of Haskell's type system enforces well-formedness in the structure of GSN graphs, but there are other errors that can violate well-formedness. 
\jitcert provides functions to check for these well-formedness violations (module \texttt{JitCert.Analysis}).

\subsection{Unbound node references}
Nodes are introduced into scope for the node's parent and the parent's descendents (\Cref{subsec:contextnodes}). 
A noun context node can only be referenced when it is in scope. 
\jitcert provides the \texttt{checkUnbound} function to check when this invariant is violated.


\begin{figure}
\centering
\includegraphics[scale=0.5]{img/figure6.png}
\caption{Example of GSN with references to context nodes.}
\label{fig:fig6}
\end{figure}


For example, consider the GSN shown in \Cref{fig:fig6}. 
The \texttt{"A goal"} node introduces the program \texttt{p} as a context node.
Only descendants of the \texttt{"A goal"} node can reference \texttt{p}, but the cousin goal has a reference to \texttt{p}.
This is a violation. 
When \texttt{checkUnbound} is run on this GSN, 
it produces the error \texttt{"Unbound error at node: Cousin goal that program, p is safe."}



% Describe issue. Show example. Describe function that performs check for this error.
\subsection{Values assigned to unbound node references}
It is possible to assign values to noun context nodes and solution nodes even if the nodes are not in scope. 
\jitcert provides the \texttt{checkUnboundValues} to check when this happens. 

\Cref{fig:fig7} extends the example in \Cref{fig:fig5} by using it as a subgoal. 
In addition, it creates another subgoal that the program uses a
sufficiently random random number generator. 
A context node \texttt{rng} is added for the random number generator. 
\begin{mcode}
rng <- context @RNG "g" Nothing
\end{mcode}
\texttt{rng} is not in scope for the functions are correct goal node. 
We can assign \texttt{rng} a value in the nested scope for this node.
\begin{code}
setContextValuesWith fs functions $ \(function, generator) -> 
    setContextValue rng generator
\end{code}
Now when we run \texttt{checkUnboundValues}, we get the error 
\texttt{"Value assigned to node, but node is not in scope: rng, g."}

\begin{figure}
\centering
\makebox[\textwidth][c]{\includegraphics[scale=0.5]{img/figure7.png}}
\caption{Example of GSN with references to context nodes.}
\label{fig:fig7}
\end{figure}

\subsection{Shadowed value assignments}
Since \jitcert supports nested scopes, it is possible to define a value for a node in one environment, and then define it again in a nested child environment. 
This is called shadowing value assignments. 
While shadowing value assignments is technically valid, it does not typically make sense to do so in practice.
\jitcert provides the \texttt{checkShadowedValues} to check for shadowed values. 

Consider the example GSN in \Cref{fig:fig5} (from \Cref{subsec:nestedenv}).
Previously, we assigned values to the nodes \texttt{f}, \texttt{t}, and \texttt{s} in the nested environment. 
\begin{code}
setContextValuesWith fs functions $ lambda(function, functionTest, functionTestResult) -> do
    setContextValue f function
    setContextValue t functionTest
    setSolutionEvidence s functionTestResult

    return function
\end{code}
We can shadow value assigned to node \texttt{t} by setting its value in the top level environment. 
\begin{mcode}
setContextValue t (Test "test_constant" "32dfe08c")
\end{mcode}
When \texttt{checkShadowedValues} is run on this GSN, 
it produces the error \texttt{"Value is shadowed at node: test, t."}

% TODO: PolicyWhen must have 1 child as the context verb predicate + vice versa
% Unused, never referenced nodes.
% Acyclic check

\section{Patterns}

There are common patterns that frequently appear in many GSNs. 
\jitcert provides functionality to easily create and identify these patterns. 

\subsection{Contexts}

There are many types that are frequently used for noun contexts in GSNs. 
\jitcert defines some of these types in the \texttt{JitCert.Context} module. 

For example, files are a common value for noun contexts. 
\jitcert defines the following data type for files. 
\begin{code}
data File = File {
      fileName :: FilePath
    , fileDirectory :: FilePath
    , fileContent :: Hash
    }
\end{code}
\jitcert implements a \texttt{RenderContext} instance for \texttt{File} so that it may be used in noun context nodes.
In addition, the function, \texttt{loadFile :: FilePath -> IO File}, is offered as an easy way to create a \texttt{File} given its filepath. 


\subsection{Evidence}

In a similar manner to contexts, \jitcert defines types for commonly used evidence types in the module \texttt{JitCert.Evidence}. 
Manual inspection is a common solution. 
The evidence for inspection is represented with the \texttt{Inspected} type.
\begin{mcode}
data Inspected c p = Inspected {
      inspector :: T.Text
    , inspectionDate :: UTCTime
    , inspectionPasses :: Bool
    }
\end{mcode}
\texttt{Inspected} serves as evidence by denoting who did the inspection, when the inspection took place, and the result of the inspection. 
% \jitcert provides an \texttt{Evidence} instance for \texttt{Inspected}.

\subsection{Combinators}

\jitcert has combinators to facilitate the easy creation of common patterns (module \texttt{JitCert.Combinators}). 
One such pattern is that a property is either supported by manual inspection or static analysis. 
The function \texttt{manualInspectionOrStaticAnalysis} creates a GSN with this pattern and has the following type.
\begin{mcode}
manualInspectionOrStaticAnalysis :: (Property p, Monad m) => Sentence -> Node c Context -> p -> BuilderT m (Node d Goal, Node (Inspected c p) Solution, Node (StaticProgramAnalysisCertificate c p) Solution)
\end{mcode}
It takes a sentence for the goal node's label, a context node, and a property as arguments. 
Here's an example that argues a program is safe by manual inspection or static analysis. 
\begin{mcode}
p <- context @SoftwareProgram "p" Nothing

(g, _, _) <- manualInspectionOrStaticAnalysis [SFContext p, SFProperty IsSafe] p IsSafe
addEdge g p
\end{mcode}
This results in the GSN shown in \Cref{fig:fig8}.
Notice that the GSN substructure is automatically created by the \texttt{manualInspectionOrStaticAnalysis} combinator. 

\begin{figure}
\centering
\makebox[\textwidth][c]{\includegraphics[scale=0.5]{img/figure8.png}}
\caption{Simple GSN example.}
\label{fig:fig8}
\end{figure}

\paragraph{User-defined combinators} While JitCert provides built-in combinators for common GSN structures, it is simple for users to define their own combinators to reuse argument structures in their GSN. 
Users can do so by defining a function that creates their argument structure and returns references to relevant nodes. 
For example, we can convert \Cref{fig:fig2} into a user-defined combinator by creating the function \texttt{testFunction}. 
\begin{mcode}
testFunction :: Node g Goal -> BuilderT m (Node Function Context, 
    Node Test Context, Node (Tests Function IsCorrect) Solution)
testFunction parent = do
  f <- context @Function "f" Nothing
  t <- context @Test "t" Nothing

  g <- goal PolicyAnd [SFContext f, SFText "is correct"]
  addEdge g f
  addEdge g t

  s <- solution (Tests t f IsCorrect)
  addEdge g s

  addEdge parent g
  return (f, t, s)
\end{mcode}
The majority of the function body of \texttt{testFunction} is the same as the code for \Cref{fig:fig2}. 
The only differences are on the last two lines. 
An edge is added from the \texttt{parent} goal node, which is provided as an argument. 
The function also returns references to the function context node, test context node, and testing solution node.
Now the user can easily reuse this pattern by calling \texttt{testFunction}. 


\subsection{Queries}
\label{subsec:patqueries}

\jitcert offers querying functions to identify patterns (\texttt{JitCert.Query}).
To find goals supported by manual inspection or static analysis, 
we can use the 
\texttt{findManualInspectionOrStaticAnalysis} function. 
Given a GSN graph, it returns the list of nodes that are goal nodes with an 
\texttt{or} policy and child solution nodes of
manual inspection and static analysis. 
% solution nodes.

Users can write their own queries with the \texttt{findNodes} function. 
\begin{code}
findNodes :: GSNGraph -> (GSNGraph -> SomeNode -> ParentNodes -> ChildNodes -> Bool) -> [SomeNode]
\end{code}
Users can specify a predicate function of whether a node matches a pattern. 
\texttt{findNodes} will then search the GSN graph for all nodes that match the user defined pattern.

\section{Queries}
\label{sec:queries}

\subsection{Pattern queries}

As previously discussed in \Cref{subsec:patqueries}, \jitcert allows users to run queries on the structure of GSNs and identify patterns. 
\jitcert also enables additional types of queries that are discussed in the rest of this section. 

\subsection{Environment queries}

\jitcert provides functions to query GSN structures and their environments. 
For example, if we want to identify all the context and solution nodes that are missing values and evidence, 
we can use the \texttt{findMissingValues} function.
\begin{code}
findMissingValues :: GSNGraph -> GSNEnv -> [SomeNode]
\end{code}
This returns a list of the nodes that are missing values, making it easy for certifiers to discover what is still needed to support the GSN argument. 

Users can write their own queries with the \texttt{findNodesWithEnv} function. 
\begin{code}
findNodesWithEnv :: GSNGraph -> GSNEnv -> (GSNGraph -> GSNEnv -> SomeNode -> ParentNodes -> ChildNodes -> Bool) -> [SomeNode]
\end{code}
This is similar to the \texttt{findNodes} function, except it also takes an environment as an argument to the predicate.

\jitcert provides two functions that returns a list of context and evidence values for a given context or solution node. 
\begin{code}
lookupContextNodeValues :: NodeId c Context -> GSNGraph -> GSNEnv
    -> [SomeContext]
lookupSolutionNodeValues :: NodeId c Solution -> GSNEnv
    -> [SomeEvidence]
\end{code}
For example, calling \texttt{lookupContextNodeValues} on \texttt{fs} from \Cref{fig:fig5} returns the \emph{fibonacci} and \emph{foo} functions. 
On the other hand, calling \texttt{lookupSolutionNodeValues} on the figure's solution node returns the passing \texttt{TestResult} for \emph{fibonacci}
and the failing \texttt{TestResult} for \emph{foo}.

\subsection{Difference queries}

\begin{figure}
\centering
\makebox[\textwidth][c]{\includegraphics[scale=0.5]{img/figure9.png}}
\caption{Simple GSN example.}
\label{fig:fig9}
\end{figure}

\begin{figure}
\centering
\makebox[\textwidth][c]{\includegraphics[scale=0.5]{img/figure10.png}}
\caption{Simple GSN example.}
\label{fig:fig10}
\end{figure}

Oftentimes when recertifying a system due to a new release, parts of the system change, but the GSN structure remains the same. 
To speed up the recertification process, it is helpful to identify which parts of a GSN argument are impacted by the changes to the system and need to be rechecked. 
\jitcert provides the \texttt{envDiff} function to identify which parts of a GSN are impacted by system changes.
\begin{code}
envDiff :: GSNGraph -> GSNEnv -> GSNEnv -> Set SomeNode
\end{code}
The first argument is the GSN graph structure. 
The second argument is the initial GSN environment where values are supplied to context and solution nodes. 
The third argument is the updated GSN environment where context and solution node values may differ.
\texttt{envDiff} returns the set of nodes where its value or any of its dependencies' values have changed. 
These nodes of the GSN argument are impacted by the changes to the system and need to be recertified. 
It is important to note that for this functionality to be correct, dependencies must be correctly encoded in the GSN graph. 


As an example, consider the GSN from \Cref{fig:fig7} where the program uses \emph{Dual\_EC\_DRBG} as its random number generator (\Cref{fig:fig9}). 
% https://en.wikipedia.org/wiki/NIST_SP_800-90A
After \emph{NIST SP 800-90A} removed \emph{Dual\_EC\_DRBG} as a recommended random number generator, 
the program switched to using \emph{HMAC\_DRBG}. 
The updated GSN is shown in \Cref{fig:fig10}, which highlights the argument nodes dependent on \texttt{g} that need to be recertified. 



\section{Web interface}

\jitcert offers a web interface to view and run queries on GSNs. 
\Cref{fig:web1} is a screenshot of the web interface rendering the GSN from \Cref{fig:fig5}. 
The options sidebar allows the user to specify which environment of the GSN to load and what queries to run. 
Queries from \Cref{sec:queries} like \texttt{checkUnbound}, \texttt{checkUnboundValues}, and \texttt{checkShadowedValues} can be run where 
resulting nodes are highlighted and error messages shown. 

The GSN can be normalized with 
\texttt{normalizeGSN} (\Cref{subsec:normalize}). 

When enabled, the interface will highlight the immediate dependencies of a selected node. 
These are nodes that have a reference to the selected node. 
This is implemented internally with the \texttt{buildReferencedDependencyGraph} function. 

The difference between two environments can be highlighted with the \texttt{envDiff} function (\Cref{sec:queries}). 

The details sidebar shows more information about a selected node, like its values from the environment (\Cref{fig:web2}). 


\begin{figure}
\centering
\makebox[\textwidth][c]{\includegraphics[scale=0.5]{img/web1.png}}
\caption{\jitcert web interface.}
\label{fig:web1}
\end{figure}

\begin{figure}
\centering
\makebox[\textwidth][c]{\includegraphics[scale=0.5]{img/web2.png}}
\caption{\jitcert web interface with details for node \texttt{t}.}
\label{fig:web2}
\end{figure}




\bibliographystyle{plain}
\bibliography{refs}

\end{document}
